\chapter{Planteamiento de la investigación}
Es la identificación de la problemática que se trata de solucionar por medio de la investigación y, para la tesis, es en sí la elección del tema que servirá de base para elaborarla mediante una proposición concreta indicando el contexto preliminar del problema, definición del problema, objetivos, justificación/relevancia. 

El problema deberá cumplir una serie de condiciones que de alguna forma justifiquen el esfuerzo necesario para resolverlo. Entre ellas: originalidad, trascendencia, actualidad, relevancia y la posibilidad de permitir el uso de lo aprendido a lo largo de la carrera.

\section{Planteamiento del problema}
Descripción del contexto y definición del problema de investigación y, para la tesis, es en sí la elección del tema que servirá de base para elaborarla mediante una proposición concreta en la se contemple lo siguiente:

\begin{itemize}
    \item Identificación de los hechos que afectan al problema 
    \item Descubrimiento de las causas y efectos del problema 
    \item Planteamiento general de la problemática 
    \item Formulación del medio en el que se desarrolla el problema 
\end{itemize}

El problema deberá cumplir una serie de condiciones que de alguna forma justifiquen el esfuerzo necesario para resolverlo. Entre ellas: originalidad, trascendencia, actualidad, relevancia y la posibilidad de permitir el uso de lo aprendido a lo largo de la carrera. Se describe la situación problemática del contexto desde la perspectiva científica. Debe enunciarse referencias que sustenten la problemática 

\section{Objetivos de la Investigación}
Es la definición de lo que se pretende cumplir con la tesis. Debe existir una estricta correspondencia entre los objetivos, el planteamiento del problema y las conclusiones. También demanda una redacción sencilla, concreta y que contemple las siguientes reglas:

\begin{itemize}
    \item Iniciar el objetivo con un verbo en infinitivo.
    \item Determinar primero el qué se quiere y después el para qué se hace.
    \item Limitar la redacción a frases sustantivas.
\end{itemize}

%% La autoindentación dentro de subsecciones no es posible por lo que estas se manejan como ambientes separados.
\begin{subseccion}{Objetivo General}
Descripción de la finalidad principal que persigue la investigación, el motivo que le dará vigencia. El objetivo general y las preguntas de investigación están íntimamente relacionados entre si, por lo que deben ser coherentes
\end{subseccion}

\begin{subseccion}{Objetivos Específicos}
Señalan las actividades que se deben cumplir para avanzar en la investigación y lo que se pretende lograr en cada una de las etapas de ella, por ende, la suma de los resultados de cada uno de los objetivos específicos permitirá alcanzar el propósito integral del objetivo general. Especifica los logros concatenados que se pretende conseguir.
Para la formulación de los objetivos considere lo siguiente:

\begin{itemize}
    \item Deben estar dirigidos a los elementos básicos del problema
    \item Deben ser medibles y observables
    \item Deben ser claros y precisos
    \item Su formulación debe involucrar resultados concretos
    \item El alcance de los objetivos debe estar dentro de las posibilidades del investigador
    \item Deben ser expresados en verbos en infinitivos
\end{itemize}
\end{subseccion}

\section{Tipo y Nivel de Investigación}
Se debe de indicar a qué tipo de investigación corresponde el trabajo que se está realizando. Fundamentar igualmente el tipo que se ha especificado. A continuación, se describen los diferentes tipos.\newline 

Las investigaciones se dividen según el objetivo de investigación en dos tipos:\newline

\begin{description}

    \item[Investigación básica] Estudio e investigación que está destinado a aumentar nuestra base de conocimientos científicos. Este tipo de investigación a menudo es puramente teórica con la intención de aumentar nuestra comprensión de ciertos fenómenos o comportamientos, pero no trata de resolver o tratar estos problemas. Es una investigación sistemática, controlada, empírica y crítica de propuestas hipotéticas, exigiendo su aprobación.
    
    \item[Investigación aplicada] Estudio e investigación que está destinado a tomar acciones  con la finalidad de resolver problemas.
    
\end{description}

Según los datos que se vayan a utilizar, los tipos de investigación pueden ser:

\begin{description}

    \item[Investigación cualitativa] es una investigación que obtiene datos que no son cuantificables, por lo que se obtiene gran cantidad de información, propenso a ser subjetivo y no presenta una justificación del asunto a investigar. Suele realizarse en una etapa inicial para posteriormente ser completado de forma más objetiva.
    
    \item[Investigación cuantitativa] es una investigación objetiva ya que se establecen mediciones reales obteniendo así una cantidad de datos fiables, permitiendo realizar explicaciones contrastadas, realizar estadísticas, etc.

\end{description}

Se debe de indicar el nivel de investigación correspondiente al trabajo que se está realizando. Fundamentar igualmente el nivel que se ha especificado. Una investigación puede iniciarse como explorativa, luego ser descriptiva y terminar como explicativa. Pero, con cual de ellos se indica una investigación, esto depende de dos factores: el estado de conocimiento en el tema de investigación que nos revele la revisión de la literatura y el enfoque que el investigador pretende dar a su trabajo de investigación.  A continuación se describen los diferentes niveles.\newline

Considerando la investigación en base al nivel de profundización, estos se clasifican en:

\begin{description}

    \item[Investigación explorativa:] Los datos e analizan de forma poca profunda con la intención de tener una idea para posteriormente ser ampliados. En este caso, se conoce poco del área de estudio.

    \item[Investigación descriptiva:] Se consigue una profundización media que ayude a obtener una cantidad de datos relacionados con la investigación. Se realiza una descripción del asunto de investigación considerando circunstancias temporales y geográficas, si viene al caso
    
    \item[Investigación explicativa:] Se profundiza aún más en la investigación, ya que aparte de obtener los datos necesarios para la investigación, se hallan también las causas y consecuencias
    
    \item[Investigación correlacional:] Tiene como finalidad conocer la relación o grado de asociación entre los conceptos o variables. Permite realizar predicciones y cuantifica las relaciones entre los conceptos y variables

\end{description}

\section{Preguntas de Investigación}
Una vez que se tiene bien claro el problema, se redactan las preguntas de investigación de acuerdo al problema que se analizará. La pregunta de investigación es uno de los primeros pasos metodológicos que un investigador debe llevar a cabo cuando emprende una investigación.
La pregunta de investigación debe ser formulada de manera precisa y clara, de tal manera que no exista ambigüedad respecto al tipo de respuesta esperado.\newline

Las preguntas de investigación son operaciones mentales que hace el investigador para reconocer los puntos que le interesa abordar en su investigación. Por lo tanto, cuando las preguntas están planteadas incorrectamente el razonamiento lógico no entiende cuál es la operación que debe realizar. Las preguntas de investigación deben contener las siguientes características:

\begin{enumerate}
    \item Ser concretas: es decir no dar cabida a la vaguedad. Vaguedad significa que no se entiende exactamente por qué cosa pregunta.
    \item Ser claras: es decir dejar evidente lo que se pregunta.
    \item Ser precisas: es decir puntuales y exactas en lo que preguntan.
    \item Estar completas, es decir sobre todo que contengan sujeto o predicado.
    \item Siempre deben contener un adverbio de pregunta
\end{enumerate}

La pregunta de investigación puede ser una afirmación o un interrogante acerca del fenómeno, en forma precisa y clara, de tal forma que de ésta se desprendan los métodos, procedimientos e instrumentos.\newline

Considere que No todas las investigaciones tienen hipótesis; todo depende del grado de conocimiento sobre el problema que se investiga. Sólo necesitan hipótesis las investigaciones que ya han rebasado la fase exploratoria y se encuentran en fase confirmatoria o verificatoria. Las hipótesis, son justamente el objeto de la confirmación o verificación. Intentar forzar la presencia de hipótesis cuando el conocimiento sobre un problema o la propia naturaleza de dicho problema no lo consienten es uno de los errores más frecuentes que se comenten en la práctica.

\section{Hipótesis}
Opcional en función a la naturaleza del problema. Considerar la formulación de las hipótesis nula y alternativa.

\section{Variables}
Opcional en función a la formulación de la hipótesis. Considerar la definición de las variables independientes y dependientes.

\section{Justificación}
Expone de manera lógica aspectos como:

\begin{itemize}
    \item Importancia de la investigación.
    \item Conveniencia del estudio.
    \item Aportes/beneficios al dominio.
    \item Implicación práctica.
    \item Utilidad metodológica.
\end{itemize}

\section{Descripción de la Propuesta}
Describir claramente la propuesta del trabajo de investigación de tal forma que sea congruente con lo indicado en los objetivos y las preguntas de investigación
