\documentclass[12]{plan_tesis}

\usepackage[spanish]{babel}
\usepackage[utf8]{inputenc}
\usepackage[a4paper, total={15.5cm, 24.7cm}]{geometry}

\usepackage{appendix}
\usepackage{graphicx}
\graphicspath{ {images/} }

\usepackage{xcolor}

\title{[TITULO DEL  INFORME DE INVESTIGACIÓN]}
\author{WardoPo}
\date{March 2021}

\begin{document}

\input{portada}

\begin{resumen}
El resumen posee un conjunto de elementos los cuales dan una visión clara del trabajo descrito en no más de 120 palabras en un solo párrafo.  

La forma de redacción es en pasado impersonal, por ejemplo: una vez que recabamos la información (esta forma es incorrecta), en vez de ello se debe decir: una vez que se recabó la información (estas la forma correcta).  
\end{resumen}

\chapter{Planteamiento de la investigación}
Es la identificación de la problemática que se trata de solucionar por medio de la investigación y, para la tesis, es en sí la elección del tema que servirá de base para elaborarla mediante una proposición concreta indicando el contexto preliminar del problema, definición del problema, objetivos, justificación/relevancia. 

El problema deberá cumplir una serie de condiciones que de alguna forma justifiquen el esfuerzo necesario para resolverlo. Entre ellas: originalidad, trascendencia, actualidad, relevancia y la posibilidad de permitir el uso de lo aprendido a lo largo de la carrera.

\section{Planteamiento del problema}
Descripción del contexto y definición del problema de investigación y, para la tesis, es en sí la elección del tema que servirá de base para elaborarla mediante una proposición concreta en la se contemple lo siguiente:

\begin{itemize}
    \item Identificación de los hechos que afectan al problema 
    \item Descubrimiento de las causas y efectos del problema 
    \item Planteamiento general de la problemática 
    \item Formulación del medio en el que se desarrolla el problema 
\end{itemize}

El problema deberá cumplir una serie de condiciones que de alguna forma justifiquen el esfuerzo necesario para resolverlo. Entre ellas: originalidad, trascendencia, actualidad, relevancia y la posibilidad de permitir el uso de lo aprendido a lo largo de la carrera. Se describe la situación problemática del contexto desde la perspectiva científica. Debe enunciarse referencias que sustenten la problemática 

\section{Objetivos de la Investigación}
Es la definición de lo que se pretende cumplir con la tesis. Debe existir una estricta correspondencia entre los objetivos, el planteamiento del problema y las conclusiones. También demanda una redacción sencilla, concreta y que contemple las siguientes reglas:

\begin{itemize}
    \item Iniciar el objetivo con un verbo en infinitivo.
    \item Determinar primero el qué se quiere y después el para qué se hace.
    \item Limitar la redacción a frases sustantivas.
\end{itemize}

%% La autoindentación dentro de subsecciones no es posible por lo que estas se manejan como ambientes separados.
\begin{subseccion}{Objetivo General}
Descripción de la finalidad principal que persigue la investigación, el motivo que le dará vigencia. El objetivo general y las preguntas de investigación están íntimamente relacionados entre si, por lo que deben ser coherentes
\end{subseccion}

\begin{subseccion}{Objetivos Específicos}
Señalan las actividades que se deben cumplir para avanzar en la investigación y lo que se pretende lograr en cada una de las etapas de ella, por ende, la suma de los resultados de cada uno de los objetivos específicos permitirá alcanzar el propósito integral del objetivo general. Especifica los logros concatenados que se pretende conseguir.
Para la formulación de los objetivos considere lo siguiente:

\begin{itemize}
    \item Deben estar dirigidos a los elementos básicos del problema
    \item Deben ser medibles y observables
    \item Deben ser claros y precisos
    \item Su formulación debe involucrar resultados concretos
    \item El alcance de los objetivos debe estar dentro de las posibilidades del investigador
    \item Deben ser expresados en verbos en infinitivos
\end{itemize}
\end{subseccion}

\section{Tipo y Nivel de Investigación}
Se debe de indicar a qué tipo de investigación corresponde el trabajo que se está realizando. Fundamentar igualmente el tipo que se ha especificado. A continuación, se describen los diferentes tipos. 

Las investigaciones se dividen según el objetivo de investigación en dos tipos:\newline

\textcolor{red}{No se incluye el segmento siguiente por discrepancias en el formato original}

\section{Preguntas de Investigación}
Una vez que se tiene bien claro el problema, se redactan las preguntas de investigación de acuerdo al problema que se analizará. La pregunta de investigación es uno de los primeros pasos metodológicos que un investigador debe llevar a cabo cuando emprende una investigación.
La pregunta de investigación debe ser formulada de manera precisa y clara, de tal manera que no exista ambigüedad respecto al tipo de respuesta esperado.\newline

Las preguntas de investigación son operaciones mentales que hace el investigador para reconocer los puntos que le interesa abordar en su investigación. Por lo tanto, cuando las preguntas están planteadas incorrectamente el razonamiento lógico no entiende cuál es la operación que debe realizar. Las preguntas de investigación deben contener las siguientes características:

\begin{enumerate}
    \item Ser concretas: es decir no dar cabida a la vaguedad. Vaguedad significa que no se entiende exactamente por qué cosa pregunta.
    \item Ser claras: es decir dejar evidente lo que se pregunta.
    \item Ser precisas: es decir puntuales y exactas en lo que preguntan.
    \item Estar completas, es decir sobre todo que contengan sujeto o predicado.
    \item Siempre deben contener un adverbio de pregunta
\end{enumerate}

La pregunta de investigación puede ser una afirmación o un interrogante acerca del fenómeno, en forma precisa y clara, de tal forma que de ésta se desprendan los métodos, procedimientos e instrumentos.\newline

Considere que No todas las investigaciones tienen hipótesis; todo depende del grado de conocimiento sobre el problema que se investiga. Sólo necesitan hipótesis las investigaciones que ya han rebasado la fase exploratoria y se encuentran en fase confirmatoria o verificatoria. Las hipótesis, son justamente el objeto de la confirmación o verificación. Intentar forzar la presencia de hipótesis cuando el conocimiento sobre un problema o la propia naturaleza de dicho problema no lo consienten es uno de los errores más frecuentes que se comenten en la práctica.

\section{Hipótesis}
Opcional en función a la naturaleza del problema. Considerar la formulación de las hipótesis nula y alternativa.

\section{Variables}
Opcional en función a la formulación de la hipótesis. Considerar la definición de las variables independientes y dependientes.

\section{Justificación}
Expone de manera lógica aspectos como:

\begin{itemize}
    \item Importancia de la investigación.
    \item Conveniencia del estudio.
    \item Aportes/beneficios al dominio.
    \item Implicación práctica.
    \item Utilidad metodológica.
\end{itemize}

\section{Descripción de la Propuesta}
Describir claramente la propuesta del trabajo de investigación de tal forma que sea congruente con lo indicado en los objetivos y las preguntas de investigación

\chapter{Revisión y Fundamentación Teórica}

\section{Estado del Arte}
En este apartado, considerado también como marco de referencia o estado del arte, se deberá analizar todo aquello que se ha escrito acerca del objeto de estudio: ¿qué se sabe del tema? ¿qué estudios se han hecho en relación a él? ¿desde qué perspectivas se ha abordado?\newline

Se debe evitar ahondar en teorías que sólo planteen un solo aspecto del fenómeno.\newline

Las funciones de los antecedentes son:

\begin{itemize}
    \item Delimitar el área de investigación
    \item Hacer un compendio de estudios realizados relacionados al tema de investigación
    \item Ayudar a prevenir errores que se han cometido en otros estudios
    \item Orientar sobre cómo habrá de llevarse a cabo el estudio
    \item Proveer un marco de referencia para interpretar los resultados del estudio
\end{itemize}

Esto implica realizar una revisión crítica de la literatura correspondiente, pertinente y actualizada. Al final, es importante fijar una determinada postura ante el fenómeno en cuestión.

\section{Fundamentos teóricos}


\chapter{Diseño Metodológico}

Es la descripción de la metodología utilizada en el transcurso de la investigación detallando, lo más posible, uno a uno los métodos, técnicas, procedimientos y demás herramientas que sirvieron de alguna manera para realizar el trabajo de tesis. 
La metodología aclara, en forma muy detallada, los pasos y procedimientos utilizados para llevar a cabo la investigación. Debe de quedar muy claro cómo se realizó la investigación. La metodología debe escribirse en pasado.

\section{Sujetos/unidades de estudio}
\section{Universo, Población y Muestra }
\section{Técnicas/instrumentos para la recolección de los datos}
\section{Técnicas para el análisis de los datos}

\chapter{Plan de Trabajo y Cronograma}

Especificar los tiempos a emplear en cada etapa del proceso del trabajo de investigación para ordenar las actividades y programarlas. Las actividades deben ir numeradas y debe presentarse a través de un Diagrama de Gantt. Este cronograma sirve para controlar y monitorizar el desarrollo de la investigación. \cite{einstein}


\bibliographystyle{IEEEtran}
\bibliography{bibliografia}

\appendix
\chapter{Apendice 1: Ejemplo}
Aqui va el contenido del apéndice
\end{document}
